\section{Conclusion} \label{sec:conclusion}

In this paper, we analyze the impact of write pulse width on area, performance, and energy of STT-RAM array and develop a methodology for device-architecture co-design of STT-RAM based caches with different optimization goals. We take both near-commercialized in-plane MTJ and advanced PMTJ as optimization targets. Our study shows that for a given MTJ spec the quality of STT-RAM macro strongly depends on the write pulse width. In general, reducing write pulse width will harm area, read operation and leakage and these metrics were exacerbated when the write pulse is shorter than some certain width in processional mode. While write latency/energy is not a non-monotonic function of write pulse width. Therefore it's important to find the optimal write pulse width for minimum write latency or energy. We combine the write pulse width optimization with other architectural techniques to design a 64MB STT-RAM chip and three 16KB STT-RAM based caches. The STT-RAM caches are verified as L1 cache replacement in an embedded system and the simulation results show that by utilizing advanced PMTJ STT-RAM based L1 cache can outperform SRAM in system performance and energy separately or even simultaneously. 