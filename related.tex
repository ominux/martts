\section{Related Work} \label{sec:relate}

Many modeling tools have been developed to enable system-level design exploration for SRAM- or DRAM-based cache and memory design.  For example, CACTI~\cite{CACTI51} is a tool that has been widely used in the computer architecture community to estimate the speed, power, and area of SRAM and DRAM caches.  Evans and Franzon~\cite{CACTI:JSSC95:Evans} developed an energy model for SRAMs and used it to predict an optimum organization for caches.  eCACTI~\cite{eCACTI} incorporated a leakage power model into CACTI.  Muralimanohar \emph{et al.}~\cite{CACTI60} modeled large capacity caches through the use of an interconnect-centric organization composed of mats and request/reply H-tree networks. In addition, CACTI has also been extended to evaluate the performance, power, and area for STT-RAM~\cite{CACTI:DAC08:Dong}, PCRAM~\cite{CACTI:PCRAMsim}, NAND flash~\cite{CACTI:DATE10:Mohan}, and ReRAM~\cite{CACTI:DATE11:Xu}. However, fixed write pulse width were assumed in all the the mentioned work. In our work we developed a system-level modeling of STT-RAM with varied write pulse width coupling corresponding write current and integrated the model in a tool called NVsim~\cite{CACTI:PCRAMsim}, which is a circuit-Level performance, energy, and area simulator for emerging non-volatile memories.

There have seen several work on design methodology for STT-RAM from both circuit and architecture perspective. Li \emph{et al.} developed a physics-based MTJ model and their analysis results showed that the sizing of access NMOS transistor has critical impact on the stability and the density of STT-RAM. Chatterjee \emph{et al.} had a more  through study on co-designing the sizing of the access transitor and operating voltage to achieve minimum energy dissipation. Moreover, Smullen \emph{et al.} illustrated STT-RAM cell design for optimizing read latency and write latency separately in the presence of clock cycles. Similarly, Xu \emph{et al.} quantitatively analyze the impact of write latency and read latency trade-off of STT-RAM on system performance. However, most of these work were using STT-RAM as a last-level cache replacement and few of them gave a comprehensive study on how to design a STT-RAM macro with minimum write latency or write energy by choosing the optimal write pulse width.

The contributions of this work are listed as follow,
\begin{itemize}
\item We provide a detailed analysis of the impact of write pulse width on area, read latency/enery, write latency/energy of STT-RAM with different capacities. Our results indicate that the write pulse width optimizing STT-RAM write latency or energy will increase as the cache capacity increase.
\item To the best of our knowledge, we are the first to explore the design space of STT-RAM using advanced perpendicular MTJs. Surprisingly we find that by utilizing such MTJ spec to build L1 cache with the optimization methodology we developed, STT-RAM can have competitive write latency/energy as SRAM while maintaining equal or better read latency/energy than SRAM.
\end{itemize} 