\section{Introduction} \label{sec:intro}

Experiments have been performed in device-level research in order to operate a MTJ (magnetic tunnel junction) at minimum energy or EDP (energy-delay-product) by applying write voltage with varied pulse width on MTJ. However, the optimal operating write pulse width from cell-level point of view is not necessarily the best operating point from system-level point of view. Normally an STT-RAM memory cell consists of an access transistor in serial with a MTJ. Short write pulse induced large switching current requires large access transistor for providing enough driving current, which consequently brings more circuit design challenges of the STT-RAM prototype. Specifically, it worsens the area, latency, dynamic energy and leakage power of both memory cells and peripheral circuitry.  Thus it's imperative to offer a methodology for system-level analysis of the memory macro to quantitatively address the trade of performance and energy of STT-RAM. The most attractive part of this work is that we will present the analysis results from architecture-level simulation, which takes the parameters from technology side as input, to device researchers and even circuit designers to adjust their experiments and designs accordingly. The feedback mode of such device-architecture co-optimization will follow the flow several times until desired design goal is achieved.

Memristor-based RRAM is considered as the most promising universal memory technology since it has faster write latency compared to PCRAM and has smaller cell structure compared to MRAM (or STT-RAM).  More importantly, Memristor-based RRAM has the potential to build cross-point memory array without access devices because the memristor cell itself has sufficient non-linearity to differentiate the accessing mode and non-accessing mode.  However, the cross-point structure also brings extra challenges to the peripheral circuitry design.  In this work, we implement a system-level performance, energy, and area model to estimate the impact of different peripheral circuitry choices on the RRAM array design.

Many modeling tools have been developed to enable system-level design exploration for SRAM- or DRAM-based cache and memory design.  For example, CACTI~\cite{CACTI51} is a tool that has been widely used in the computer architecture community to estimate the speed, power, and area of SRAM and DRAM caches.  Evans and Franzon~\cite{CACTI:JSSC95:Evans} developed an energy model for SRAMs and used it to predict an optimum organization for caches.  eCACTI~\cite{eCACTI} incorporated a leakage power model into CACTI.  Muralimanohar \emph{et al.}~\cite{CACTI60} modeled large capacity caches through the use of an interconnect-centric organization composed of mats and request/reply H-tree networks.

In addition, CACTI has also been extended to evaluate the performance, power, and area for STT-RAM~\cite{CACTI:DAC08:Dong}, PCRAM~\cite{CACTI:PCRAMsim}, and NAND flash~\cite{CACTI:DATE10:Mohan}. However, it is not straightforward to use the CACTI-based models to estimate the performance, energy, and area for ultra high density memristor designs.  The conventional peripheral circuitry, which is usually optimized to reduce the memory access latency, would result in a memristor design with very low area efficiency and impair the benefits achieved by the minimum physical size of the memristor cells.  As a result, several new array structures, such as cross-point access, non-H-tree organization, external sensing, and minimum-sized row decoder, are proposed in this work.  By leveraging these new features, the unique property of the memristor's small cell size can be greatly exploited and an ultra high density non-volatile memory system can be implemented. 

\begin{comment}
Comment Paragraph
\end{comment}