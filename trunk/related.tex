\section{Related Work} \label{sec:relate}

Many modeling tools have been developed to enable system-level design exploration for SRAM- or DRAM-based cache and memory design.  For example, CACTI~\cite{CACTI51} is a tool that has been widely used in the computer architecture community to estimate the speed, power, and area of SRAM and DRAM caches.  Evans and Franzon~\cite{CACTI:JSSC95:Evans} developed an energy model for SRAMs and used it to predict an optimum organization for caches.  eCACTI~\cite{eCACTI} incorporated a leakage power model into CACTI.  Muralimanohar \emph{et al.}~\cite{CACTI60} modeled large capacity caches through the use of an interconnect-centric organization composed of mats and request/reply H-tree networks.

In addition, CACTI has also been extended to evaluate the performance, power, and area for STT-RAM~\cite{CACTI:DAC08:Dong}, PCRAM~\cite{CACTI:PCRAMsim}, and NAND flash~\cite{CACTI:DATE10:Mohan}. However, it is not straightforward to use the CACTI-based models to estimate the performance, energy, and area for ultra high density memristor designs.  The conventional peripheral circuitry, which is usually optimized to reduce the memory access latency, would result in a memristor design with very low area efficiency and impair the benefits achieved by the minimum physical size of the memristor cells.  As a result, several new array structures, such as cross-point access, non-H-tree organization, external sensing, and minimum-sized row decoder, are proposed in this work.  By leveraging these new features, the unique property of the memristor's small cell size can be greatly exploited and an ultra high density non-volatile memory system can be implemented. 