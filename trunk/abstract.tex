\begin{abstract}

STT-RAM (Spin-Transfer Torque Random Access Memory) is a fast,  scalable, durable non-volatile memory which can be embedded into standard CMOS process. A wide range of write speeds from 1 ns to 20 ns have been demonstrated for STT-RAM. The switching current of STT-RAM is inversely proportional to the write pulse width and it increases rapidly below 10ns. Experiments have been performed in device-level research in order to operate a MTJ (magnetic tunnel junction) at minimum energy or EDP (energy-delay-product) by applying write voltage with varied pulse width on MTJ. However, the optimal operating write pulse width from cell-level point of view is not necessarily the best operating point from system-level point of view. Normally an STT-RAM memory cell consists of an access transistor in serial with a MTJ. Short write pulse induced large switching current requires large access transistor for providing enough driving current, which consequently brings more circuit design challenges of the STT-RAM prototype. Specifically, it worsens the area, latency, dynamic energy and leakage power of both memory cells and peripheral circuitry.  Thus it's imperative to offer a methodology for system-level analysis of the memory macro to quantitatively address the trade of performance and energy of STT-RAM. The most attractive part of this work is that we will present the analysis results from architecture-level simulation, which takes the parameters from technology side as input, to device researchers and even circuit designers to adjust their experiments and designs accordingly. The feedback mode of such device-architecture co-optimization will follow the flow several times until desired design goal is achieved.

\end{abstract} 