\begin{abstract}

Spin-transfer torque random access memory (STT-RAM) is a fast, scalable, durable non-volatile memory which can be embedded into standard CMOS process. A wide range of write speeds from 1 ns to 100 ns have been reported for STT-RAM. The switching current of magnetic tunnel junction (MTJ) that is the basic building block of STT-RAM is inversely proportional to the write pulse width. In this work, we provide a detailed methodology to optimize STT-RAM design for different design goals such as read performance, write performance and write energy by leveraging the trade-off between write current and write time of MTJ. We take the typical in-plane MTJ and advanced perpendicular MTJ as our optimization targets. Our study shows that the reducing write pulse width will harm read latency and energy while there might be sweet spots of write pulse width which minimize the write energy and write latency of STT-RAM caches. It's also demonstrated that the optimal value write pulse width is dependent on both MTJ spec and STT-RAM cache capacity. The simulation results indicate that by optimizing the advanced perpendicular MTJ based cache design STT-RAM can compete against SRAM for some embedded applications.

\end{abstract} 