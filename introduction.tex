\section{Introduction} \label{sec:intro}

Memristor, a portmanteau of ``memory resistor'', is a generalized resistance that maintains a functional relationship between the time integrals of current and voltage.  In 1971 memristor was defined by Chua~\cite{memristor:chua} as the fourth fundamental circuit element from the completeness of relations between the four basic circuit variables, namely, current, voltage, charge, and flux-linkage.  In 2008, HP Labs~\cite{memristor:missing} presented the first experimental realization as well as a theoretical model of memristor.
\begin{comment}
Comment Template
\end{comment}
As rapid progress in nanotechnology has been achieved during the past few years, research has been widely explored in nanoscale bipolar RRAM built with different materials and working under different mechanisms~\cite{rram:hfo2,rram:al2o3}. These devices have demonstrated many distinctive features, including non-volatility, non-linearity, fast access, high density, and good scalability~\cite{memristor:how}.
%The target for memristors, in the long term, is to transform computing by building adaptive control circuits that learn and to enable the functionality of computation using brain-like architecture of new analog circuits.

Memristor-based RRAM is considered as the most promising universal memory technology since it has faster write latency compared to PCRAM and has smaller cell structure compared to MRAM (or STT-RAM).  More importantly, Memristor-based RRAM has the potential to build cross-point memory array without access devices because the memristor cell itself has sufficient non-linearity to differentiate the accessing mode and non-accessing mode.  However, the cross-point structure also brings extra challenges to the peripheral circuitry design.  In this work, we implement a system-level performance, energy, and area model to estimate the impact of different peripheral circuitry choices on the RRAM array design.

Many modeling tools have been developed to enable system-level design exploration for SRAM- or DRAM-based cache and memory design.  For example, CACTI~\cite{CACTI51} is a tool that has been widely used in the computer architecture community to estimate the speed, power, and area of SRAM and DRAM caches.  Evans and Franzon~\cite{CACTI:JSSC95:Evans} developed an energy model for SRAMs and used it to predict an optimum organization for caches.  eCACTI~\cite{eCACTI} incorporated a leakage power model into CACTI.  Muralimanohar \emph{et al.}~\cite{CACTI60} modeled large capacity caches through the use of an interconnect-centric organization composed of mats and request/reply H-tree networks.

In addition, CACTI has also been extended to evaluate the performance, power, and area for STT-RAM~\cite{CACTI:DAC08:Dong}, PCRAM~\cite{CACTI:PCRAMsim}, and NAND flash~\cite{CACTI:DATE10:Mohan}. However, it is not straightforward to use the CACTI-based models to estimate the performance, energy, and area for ultra high density memristor designs.  The conventional peripheral circuitry, which is usually optimized to reduce the memory access latency, would result in a memristor design with very low area efficiency and impair the benefits achieved by the minimum physical size of the memristor cells.  As a result, several new array structures, such as cross-point access, non-H-tree organization, external sensing, and minimum-sized row decoder, are proposed in this work.  By leveraging these new features, the unique property of the memristor's small cell size can be greatly exploited and an ultra high density non-volatile memory system can be implemented. 