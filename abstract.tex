\begin{abstract}

Spin-transfer torque random access memory (STT-RAM) is a fast, scalable, durable non-volatile memory which can be embedded into standard CMOS process. A wide range of write speeds from $1ns$ to $100ns$ have been reported for STT-RAM. The switching current of magnetic tunnel junction (MTJ) that is the basic building block of STT-RAM is inversely proportional to the write pulse width. In this work, we provide a detailed methodology to design STT-RAM for different optimization goals such as read performance, write performance and write energy by leveraging the trade-off between write current and write time of MTJ. We take the typical in-plane MTJ and advanced perpendicular MTJ (PMTJ) as our optimization targets. Our study shows that the reducing write pulse width will harm read latency and energy. While "sweet spots" of write pulse width which minimize the write energy or write latency of STT-RAM caches may exist, depending on MTJ species, STT-RAM capacity and I/O width. The simulation results indicate that by utilizing PMTJ, the optimized STT-RAM can compete against SRAM and DRAM as universal memory replacement in low power embedded system.

\end{abstract} 